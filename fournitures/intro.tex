\section*{Fonctionnement du document}
Ce document est un document est un document expliquant notre approche pour développer une bibliothèque d'objets graphiques UML.\\
Dans ce dossier, vous pourrez repérer diverse notations, cette introduction à pour but de vous expliquer les notations afin
que vous puissiez lire en toute sérénité.
\subsection*{Le glossaire}
Un mot dans le glossaire à une police particulière, vous pourrez savoir qu'un mot est dans le glossaire lorsque vous repérerez un mot avec la police suivante: 
\policeGlossaire{leMotDansLeGlossiare}. Si vous voyez cette police, vous pouvez donc vous référez à l'annexe \ref{glossaire} page \pageref{glossaire}.
\subsection*{Les noms de méthode, d'attribut ou de classe}
Les mots se référant à un nom présent dans le code ont une police particulière, une police type ``machine à écrire'', si vous voyez la police suivante, c'est que c'est un nom 
de méthode, d'attribut ou de classe: \texttt{uneFonction}.
\subsection*{Les notes de bas de page}
Nous utilisons régulièrement des notes de bas de pages, pour donner un acronyme pour expliquer plus en détail une notion, ces notes de bas de pages sont un numéro
en expostant, vous trouverez la note correspondante en bas de la page courante, comme ceci\footnote{Ceci est une note de bas de page}.
\subsection*{Les liens hypertext}
Dans le document, nous pouvons faire référence à un lien d'un site web, tous les liens seront donc symbolisés par une petite puce, comme ceci:\\
	$\rhd$ \url{http://monLien.fr/index.html}\\

