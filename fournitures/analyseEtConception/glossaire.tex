
\sortitem {UML}{3}{(Unified Modeling Language) Langage de mod\IeC {\'e}lisation graphique \IeC {\`a} base de pictogramme. Il est apparu dans le monde du g\IeC {\'e}nie logiciel dans le cadre de la conception orient\IeC {\'e}e objet. Ce langage est compos\IeC {\'e} de diff\IeC {\'e}rents diagrammes, allant du d\IeC {\'e}veloppement \IeC {\`a} la simple analyse des besoins.}
\sortitem {Incr\IeC {\'e}ment}{3}{Fonctionnalit\IeC {\'e} du logiciel, ayant un cycle de developpement lui \IeC {\'e}tant propre (Analyse, D\IeC {\'e}veloppement, Tests). Cette fonctionnalit\IeC {\'e} doit \IeC {\^e}tre op\IeC {\'e}rationnelle pour que l'incr\IeC {\'e}ment soit termin\IeC {\'e}. Il doit am\IeC {\'e}liorer le logiciel par rapport \IeC {\`a} l'incr\IeC {\'e}ment pr\IeC {\'e}c\IeC {\'e}dent, et ne doit pas alt\IeC {\'e}rer les fonctionnalit\IeC {\'e}s pr\IeC {\'e}c\IeC {\'e}dentes.}
\sortitem {Mod\IeC {\`e}le de d\IeC {\'e}veloppement incr\IeC {\'e}mental}{3}{M\IeC {\'e}thode de d\IeC {\'e}veloppement d'un projet dans laquel le projet final est divis\IeC {\'e} en plusieurs fonctionnalit\IeC {\'e}s appell\IeC {\'e} incr\IeC {\'e}ment qui sont d\IeC {\'e}velopp\IeC {\'e}es et ajout\IeC {\'e} au projet au fur et \IeC {\`a} mesure.}
\sortitem {Diagramme de classe}{3}{Sch\IeC {\'e}ma utilis\IeC {\'e} en g\IeC {\'e}nie logiciel pour repr\IeC {\'e}senter les classes et les interfaces des syst\IeC {\`e}mes ainsi que les diff\IeC {\'e}rentes relations entre celles-ci.}
\sortitem {Diagramme de s\IeC {\'e}quence}{3}{ Repr\IeC {\'e}sentation graphique des interactions entre les acteurs et le syst\IeC {\`e}me selon un ordre chronologique. Ce diagramme est inclus dans la partie dynamique d'UML.}
\sortitem {Diagramme de cas d'utilisation}{3}{Repr\IeC {\'e}sentation graphique permettant de d\IeC {\'e}crire les int\IeC {\'e}ractions entre les acteurs et le syst\IeC {\`e}me.}
