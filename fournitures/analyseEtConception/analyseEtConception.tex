\documentclass[12pt,a4paper,openany]{report}
%%%% JNLP 
\usepackage{lmodern}
\usepackage{xcolor}
\usepackage[utf8]{inputenc}
\usepackage[T1]{fontenc}
\usepackage[francais]{babel}
\usepackage[top=1.7cm, bottom=1.7cm, left=1.5cm, right=1.5cm]{geometry}
\usepackage{pdfpages}
\usepackage{listingsutf8}
\usepackage{fancyhdr}
\usepackage{multido}
\usepackage{amssymb}
\usepackage{tikz}
\usepackage{ifthen}
\usepackage{makeidx}
\usepackage[urlbordercolor={1 1 1}, linkbordercolor={1 1 1}, urlcolor=blue]{hyperref}
\usepackage{wrapfig}

%\newCommand{\footGauche}{} Université paul sabatier Toulouse III
\newcommand{\footCentre}{}
%\newCommand{\footDroite}{} Numéro de page
\newcommand{\premierDestinataire}{Monsieur Max \bsc{Chevalier}}
\newcommand{\rolePremierDestinataire}{Responsable projets}

\newcommand{\secondDestinaire}{Monsieur Thierry \bsc{Millan}}
\newcommand{\roleSecondDestinaire}{Client}

\newcommand{\troisiemeDestinaire}{Madame Caroline \bsc{Kross}}
\newcommand{\roleTroisiemeDestinaire}{Tutrice}

\newcommand{\quatriemeDestinaire}{}
\newcommand{\roleQuatriemeDestinaire}{}

\newcommand{\cinquiemeDestinaire}{}
\newcommand{\roleCinquiereDestinaire}{}
\newcommand{\titreDocument}{Analyse et conception}


\usepackage{verbatim}
\usepackage{datatool}

\date{\today}

\chead{}
\rhead{Projet \#20}
\lhead{Bibliothèque d'objets grahiques UML}
\makeindex
\lfoot{Université Paul Sabatier Toulouse III}
\rfoot{--~\thepage~--}
\cfoot{\footCentre}
\makeglossary
\makeatletter
\def\clap#1{\hbox to 0pt{\hss #1\hss}}%
\def\ligne#1{%
\hbox to \hsize{%
\vbox{\centering #1}}}%
\def\haut#1#2#3{%
\hbox to \hsize{%
\rlap{\vtop{\raggedright #1}}%
\hss
\clap{\vtop{\centering #2}}%
\hss
\llap{\vtop{\raggedleft #3}}}}%
\def\bas#1#2#3{%
\hbox to \hsize{%
\rlap{\vbox{\raggedright #1}}%
\hss \clap{\vbox{\centering #2}}%
\hss
\llap{\vbox{\raggedleft #3}}}}%
\def\maketitle{%
\thispagestyle{empty}\vbox to \vsize{%
\haut{}{\@blurb}{}
\begin{flushleft}
	\vspace{1cm}
	Antoine de \bsc{Roquemaurel}\\ 
	Mathieu \bsc{Soum}\\
	Geoffroy \bsc{Subias}\\
	Marie-Ly \bsc{Tang}\\
	\textit{Groupe B}\\
\end{flushleft}
\begin{flushright}
	\vspace{-3cm}
\begin{tabular}{r@{~}l}
	\ifthenelse{\equal{\premierDestinataire}{}}{
	}
	{
		Pour \premierDestinataire & (\rolePremierDestinataire) \\
	}
	\ifthenelse{\equal{\secondDestinaire}{}}{
	}
	{
		\secondDestinaire & (\roleSecondDestinaire) \\
	}
	\ifthenelse{\equal{\troisiemeDestinaire}{}}{
	}
	{
		\troisiemeDestinaire & (\roleTroisiemeDestinaire) \\
	}
	\ifthenelse{\equal{\quatriemeDestinaire}{}}{
	}
	{
		\quatriemeDestinaire & (\roleQuatriemeDestinaire) \\
	}
	\ifthenelse{\equal{\cinquiemeDestinaire}{}}{
	}
	{
		\cinquiemeDestinaire & (\roleCinquiereDestinaire) \\
	}
\end{tabular}
\end{flushright}
\vfill
\vspace{1cm}
\begin{flushleft}
\usefont{OT1}{ptm}{m}{n}
\huge \@title
\end{flushleft}
\par
\hrule height 4pt
\par
\begin{flushright}
\usefont{OT1}{phv}{m}{n}
\Large \@author
\par
\end{flushright}
\vspace{1cm}
\vfill
\vfill
\bas{}{\@location, le \@date}{}
}%
\cleardoublepage
}
\def\date#1{\def\@date{#1}}
\def\author#1{\def\@author{#1}}
\def\title#1{\def\@title{#1}}
\def\location#1{\def\@location{#1}}
\def\blurb#1{\def\@blurb{#1}}
\date{\today}
\author{}
\title{}
\location{Amiens}\blurb{}
\makeatother
\title{\titreDocument}
\author{Bibliothèque d'objets graphiques UML}

\location{Toulouse}
\blurb{%
Université Paul Sabatier -- Toulouse III\\
IUT A - Toulouse Rangueil\\
\textbf{Projet tuteuré \#20}\\[1em]
}%

\makeatletter
\newcommand{\sortitem}[3]{%
	\DTLnewrow{list}%
	\DTLnewdbentry{list}{nom}{#1}%
	\DTLnewdbentry{list}{page}{#2}%
	\DTLnewdbentry{list}{definition}{#3}%
}

\newenvironment{sortedlist}%
{%
\DTLifdbexists{list}{\DTLcleardb{list}}{\DTLnewdb{list}}%
}%
{%
\DTLsort{nom,page,definition}{list}%a
	\DTLforeach*{list}{\theNom=nom,\laPage=page,\theDefinition=definition}{%
	\paragraph{\theNom}\hspace{-8px}(p \laPage)~--~\theDefinition\hfill 
}%
}

%\newwrite{\verbatim@out@one}
%\newcommand\initiateglossary[1]{\immediate\openout \verbatim@out@one #1}

%\def\terminateglossary{\immediate\closeout\verbatim@out@one\@esphack}
%\DeclareTextFontCommand{\policeGlossaire}{\fontfamily{cmvtt}\selectfont}
\DeclareTextFontCommand{\policeGlossaire}{\fontfamily{lmss}\selectfont}
\DeclareTextFontCommand{\policePackage}{\fontfamily{phv}\selectfont}

\newwrite\glossaireVar
\openout\glossaireVar=glossaire
\write\glossaireVar{\noexpand}
\newcommand{\glo}[3]{
\policeGlossaire{\hspace{-4px}#1\hspace{-6px}}
	\write\glossaireVar{\noexpand\sortitem{#2}{\thepage}{#3}}
}

\makeatother
\newcommand{\nouveauChapitre}{ \thispagestyle{fancy} }
\def\sectionautorefname{Section}
\pagestyle{fancy}

\begin{document}
	\maketitle
	\newpage
	\tableofcontents
	\newpage
	\chapter{Analyse}
	\chapter{Conception}
	\section{Architecture générale du projet}
		L'architecture est la base d'une conception telle que nous l'avons choisie. En utilisant la notation UML, 
		nous sommes parvenu à élaborer un ``meta-modèle'' de cette notation mais en isolant uniquement l'aspect 
		graphique, faisant ainsi abstraction des multipes règles de conception que la norme UML impose. Après 
		discussion avec le client, de nombreuses modifications ont été apportées à l'archetecture d'origine 
		pour aboutir à une solution stable. Cette nouvelle architecture se compose de vingt classes, réparties 
		en quatre paquetages comme suit :
	\newline
	\begin{wrapfigure}{r}{9cm}
		\paragraph{~ ~}
		Dans cet arbre représentant notre architecture, on peut voir certains noms \textbf{en gras} ou \textit{en italique}. 
		Les noms \textbf{en gras} représentent les différents paquetages que nous avons séparés. Ceux \textit{en italiques} 
		sont des classes abstraites crées afin de factoriser le code dans l'optique de réaliser une programmation objet optimale 
		et de suivre les objectifs de propreté du code imposés par le client.\vspace{7px}
		
		Dans un premier temps, nous avons séparer les diagrammes des éléments graphiques. En effet, un diagramme sera composé de
		toute sorte d'éléments graphiques. Puis nous avons découper ces derniers en deux, isolant ainsi les lignes des elements
		de modélisation tels que les classes, les traitements ou les cas d'utilisation.\vspace{8px}
		
		\textit{ElementGraphique} et \textit{ElementModelisation} sont des classes abstraites car elle regroupe les fonctions communes
		à toues les lasses de leur package respectifs sans pour autant en fournir une implémentation de chacune d'elles -- comme par exemple la méthode qui crée la
		représentation graphique d'un élément ou celle qui permet de le supprimer d'un diagramme.
	\end{wrapfigure}
	% Arbre de classe
	\begin{itemize}
		\item \textbf{eltGraphique}
			\begin{itemize}
				\item \textit{ElementGraphique}
				\item \textbf{eltModelisation}
					\begin{itemize}
						\item \textit{Acteur}
						\item ActeurActif
						\item ActeurPassif
						\item Attribut
						\item CasUtilisation
						\item Classe
						\item \textit{ElementModelisation}
						\item Interface
						\item Methode
						\item Traitement
						\item Visibilite
						\item Variable
					\end{itemize}
				\item \textbf{ligne}
					\begin{itemize}
						\item Cardinalite
						\item Lien
						\item TypeLien
					\end{itemize}
			\end{itemize}
		\item \textbf{diagramme}
			\begin{itemize}
				\item Diagramme
				\item DiagrammeCasUtilisation
				\item DiagrammeClasse
				\item DiagrammeSequence
			\end{itemize}
	\end{itemize}\newpage 
	\paragraph{}Prenons maintenant chaque paquetage séparément 
	\subsection{Package eltGraphique}
	Le package eltGraphique regroupe toutes les classes qui représentent des éléments graphiques. Il regroupe la classe 
	\textit{ElementGraphique} et deux packages \textbf{eltModelisation} et \textbf{ligne}. Cette classe possède deux attributs 
	\texttt{graph} et \texttt{diagramme}, correspondant repectivement au graphe dans lequel sont stockés les éléments 
	et le diagramme afficher à l'écran. Elle comprend également (en plus d'un constructeur initialisant les attributs) 
	les méthodes \texttt{supprimer} permettant de supprimer un élément du graphe et du diagramme, ainsi que \texttt{creer}, 
	méthode abstraite réimplémentée dans les classes descendentes servant à creer la représentation graphique de l'objet 
	et l'afficher sur le diagramme.
	\paragraph{Package eltModelisation} Ce package regroupe toutes les classes représentant les différents éléments de 
	modélisation que l'on trouver dans les diagrammes UML de cas d'utilisation, classe et de séquence. \textit{Acteur} 
	est une classe abstraite car les acteur actif et passifs ont beaucoup de caractéristiques identiques mais n'ont pas 
	la même représentation sur un diagramme. Les classes Visibilite, Methode et Attribut permettent au client de créer 
	facilement une interface de saisie de ces éléments, facilitant l'usage du logiciel final. De plus, l'ajout de ces 
	méthodes et attributs dans des classes, des acteurs des traitements ou des interfaces pourra faciliter l'ajout futur 
	de nouvelle fonctionnalités comme apr exemple de la génération de code Java.
	\paragraph{Package ligne} Ce package regroupe peu de classe. TypeLien est une classe énumérée servant à recenser
	toutes les types graphiques de liens existant dans la notation UML.
	
\end{document}
