\chapter*{Avant-propos}
\nouveauChapitre
\addcontentsline{toc}{chapter}{Avant-propos}
\section*{Présentation du projet}
\addcontentsline{toc}{section}{Présentation du projet}
	LibUML est une 
	\glo{bibliothèque}{Bibliothèque}{Composant programmé dans un langage donné fournissant des méthodes permettant d'effectuer des tâches voulus} 
	d'objets graphiques représentant les différents éléments de modélisation de la norme 
	\glo{UML}{UML}{(Unified Modeling Language) Langage de modélisation graphique à base de pictogramme.  Il est apparu dans le monde du génie logiciel dans le cadre de la conception orientée objet. Ce langage est composé de différents diagrammes, allant du développement à la simple ana\-lyse des besoins.} 
	2\footnote{Unified Modelling Language}.
	Celle-ci a été développé dans le cadre des projets tuteurés à l'IUT\footnote{Institut Universitaire de Technologies} 'A' de Toulouse. 
	Nous l'avons développée en \glo{Java}{Java}{Langage de programmation orienté objet moderne, il compile le programme pour ensuite l'exécuter sur une machine Java, ainsi le programme une fois compilé peut être exécuté sur différentes plateformes (Windows, Linux, Mac OS X, \ldots).} 
	et conçue comme une bibliothèque pouvant être utilisée dans des programmes Java comme composant. 
	Vous pouvez vous en servir pour développer un outil complet de modélisation UML par exemple.
	\paragraph{}	
	Cette bibliothèque permet de faire différentes choses dans le cadre de la conception UML, une fois que vous aurez compris le fonctionnement de libUML vous pourrez:
	\begin{itemize}
		\item Créer des éléments de modélisation 
			\begin{itemize}
				\item Acteur actif ou passif
				\item Traitement
				\item Cas d'utilisation
				\item Classe
			\end{itemize}
		\item Relier ses composants via différents types de flèches
			\begin{itemize}
				\item Agrégation
				\item Composition
				\item Association binavigable ou mononavigable
				\item Messages synchrones ou asynchrones
				\item Généralisation
			\end{itemize}
		\item Supprimer des éléments et flèches
		\item Modifier le contenu des éléments et flèches
		\item Redimensionner les éléments de modélisation
	\end{itemize}
	\paragraph{}
Ce document a pour but de vous présenter le fonctionnement de la bibliothèque UML et de son démonstrateur. 

\section*{Présentation du groupe}
\addcontentsline{toc}{section}{Présentation du groupe}
	Notre groupe projet est composé de quatre étudiants de deuxième année de DUT Informatique à l'IUT 'A' de Toulouse, voici la composition de l'équipe: 
	\begin{itemize}
		\item Antoine de \bsc{Roquemaurel} 
		\item Mathieu \bsc{Soum} 
		\item Geoffroy \bsc{Subias}
		\item Marie-Ly \bsc{Tang} 
	\end{itemize}
	\section*{Téléchargements}
\addcontentsline{toc}{section}{Téléchargements}
	Nous vous avons envoyé deux archives zip par email, une contenant le projet Netbeans de la bibliothèque et du démonstrateur et une contenant la bibliothèque.
	\subsubsection*{Bibliothèque et démonstrateur}
	La première archive est donc un projet netbeans, cette archive contient notre bibliothèque et ses tests associés, le démonstrauteur mais
	également la bibliothèque \texttt{JGraphx} qui est indispensable au bon fonctionnement de notre bibliothèque. Nous ne garantissons le bon fonctionnement 
	de la bibliothèque qu'avec la version de \texttt{JGraphx} 1.8, celle-ci n'ayant pas été testée avec des versions différentes.

	Le projet netbeans de notre bibliothèque est également disponible en ligne, si vous souhaitez la télécharger de nouveau, elle est donc disponible à l'adresse suivante: \\
	$\rhd$ \url{http://telechargements.joohoo.fr/libUML/libUML-netbeans.zip}\\
	$\rhd$ \url{http://telechargements.joohoo.fr/libUML/JGraphX-1.8.jar}\\
	
	\subsubsection*{Documentation}
	Toute la documentation du projet est sous format \bsc{HTML}\footnote{HyperText Markup Language}, celle-ci ayant été générée grâce à Javadoc.

	Si dans ce manuel nous parlons d'une méthode et que vous n'en comprenez pas l'intéret, vous pouvez aller la chercher dans la documentation, pour chaque classe,
	le fonctionnement global de la classe est expliquée, l'utilité de chaque attribut, de chaque méthode, et ce que vous devez mettre dans les différents paramètres 
	des méthodes.  
	\paragraph{}
	Nous vous avons remis un .zip par email contenant quatre dossiers de documentation:
	\begin{itemize}
		\item La documentation privée de la \textit{bibliothèque} (Toutes les méthodes et tous les attributs)
		\item La documentation publique de la \textit{bibliothèque} (Seuls les méthodes et attributs publics ou protégés)
		\item La documentation privée du \textit{démonstrateur} (toutes les méthodes du démonstrateur) 
		\item La documentation des \textit{tests unitaires} (Toutes les méthodes de tests)
	\end{itemize}
	\paragraph{}
	Celle ci est également disponible en ligne aux adresses suivantes: \\
	$\rhd$ \url{http://documentation.joohoo.fr/libUML/bibliothequePrivee/index.html}\\
	$\rhd$ \url{http://documentation.joohoo.fr/libUML/bibliothequePublique/index.html}\\
	$\rhd$ \url{http://documentation.joohoo.fr/libUML/demonstrateurPrivee/index.html}\\\label{docDemonstrateur}
	$\rhd$ \url{http://documentation.joohoo.fr/libUML/testsUnitaires/index.html}\\
	\paragraph{}
	Vous pouvez également accéder à la documentation de \texttt{JGraphX}, cela peut vous être utile dans certains cas.\\
	$\rhd$ \url{http://documentation.joohoo.fr/JGraphX/index.html}\\ 


	\newpage
\section*{Fonctionnement du document}
\addcontentsline{toc}{section}{Fonctionnement du document}
Ce document est un document expliquant notre approche pour développer une bibliothèque d'objets graphiques UML.\\

Dans ce dossier, vous pourrez repérer diverses notations, cette partie a pour but de vous expliquer les notations afin
que vous puissiez lire en toute sérénité.
\subsubsection*{Le glossaire}
Un mot dans le glossaire a une police particulière, vous pourrez savoir qu'un mot est dans le glossaire lorsque vous repérerez un mot avec la police suivante: 
\policeGlossaire{leMotDansLeGlossiare}. Si vous voyez cette police, vous pouvez donc vous référez à l'annexe \ref{glossaire} page \pageref{glossaire}.
\subsubsection*{Les noms de méthode, d'attribut ou de classe}
Les mots se référant à un nom présent dans du code ont une police particulière, une police type ``machine à écrire'', si vous voyez la police suivante, c'est que c'est un nom 
de méthode, d'attribut ou de classe: \texttt{uneFonction}.
\subsubsection*{Les noms de paquetage}
Les noms de paquetage utiliseront une police particulière, afin que l'on puisse les différencier d'une classe ou d'une méthode, vous les trouverez 
comme suit : \policePackage{unPaquetage}.
\subsubsection*{Les notes de bas de page}
Nous utilisons régulièrement des notes de bas de pages, pour donner un acronyme, pour expliquer plus en détail une notion, ces notes de bas de pages sont un numéro
en exposant, vous trouverez la note correspondante en bas de la page courante, comme ceci\footnote{Ceci est une note de bas de page}.
\subsubsection*{Les liens hypertextes}
Dans le document, nous pouvons faire référence à un lien d'un site web, tous les liens seront donc symbolisés par une petite puce, et une police particulière comme ceci:\\
	$\rhd$ \url{http://monLien.fr/index.html}\\
	Une liste de tous les liens présents dans le document est accessible en annexe \ref{listeLiens} page \pageref{listeLiens}.

