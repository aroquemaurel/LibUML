\documentclass[12pt,a4paper,openany]{report}
\usepackage{lmodern}
\usepackage{xcolor}
\usepackage[utf8]{inputenc}
\usepackage[T1]{fontenc}
\usepackage[francais]{babel}
\usepackage[top=1.7cm, bottom=1.7cm, left=1.5cm, right=1.5cm]{geometry}
\usepackage{pdfpages}
\usepackage{listingsutf8}
\usepackage{fancyhdr}
\usepackage{multido}
\usepackage{amssymb}
\usepackage{tikz}
\usepackage{ifthen}
\usepackage{makeidx}
\usepackage[urlbordercolor={1 1 1}, linkbordercolor={1 1 1}, urlcolor=blue]{hyperref}

\newcommand{\footCentre}{}
\newcommand{\premierDestinataire}{Monsieur Thierry Millan}
\newcommand{\rolePremierDestinataire}{Client}

\newcommand{\secondDestinaire}{Madame Caroline Kross}
\newcommand{\roleSecondDestinaire}{Tutrice}

\newcommand{\troisiemeDestinaire}{}
\newcommand{\roleTroisiemeDestinaire}{}

\newcommand{\quatriemeDestinaire}{}
\newcommand{\roleQuatriemeDestinaire}{}

\newcommand{\cinquiemeDestinaire}{}
\newcommand{\roleCinquiereDestinaire}{}

\newcommand{\titreDocument}{Installation et exploitation}



\date{\today}

\chead{\headCentre}
\rhead{\headDroite}
\lhead{\headGauche}
\makeindex
\lfoot{Université Paul Sabatier Toulouse III}
\rfoot{--~\thepage~--}
\cfoot{\footCentre}
\makeglossary
\makeatletter
\def\clap#1{\hbox to 0pt{\hss #1\hss}}%
\def\ligne#1{%
\hbox to \hsize{%
\vbox{\centering #1}}}%
\def\haut#1#2#3{%
\hbox to \hsize{%
\rlap{\vtop{\raggedright #1}}%
\hss
\clap{\vtop{\centering #2}}%
\hss
\llap{\vtop{\raggedleft #3}}}}%
\def\bas#1#2#3{%
\hbox to \hsize{%
\rlap{\vbox{\raggedright #1}}%
\hss \clap{\vbox{\centering #2}}%
\hss
\llap{\vbox{\raggedleft #3}}}}%
\def\maketitle{%
\thispagestyle{empty}\vbox to \vsize{%
\haut{}{\@blurb}{}
\begin{flushleft}
	\vspace{1cm}
	Antoine de \bsc{Roquemaurel}\\ 
	Mathieu \bsc{Soum}\\
	Geoffroy \bsc{Subias}\\
	Marie-Ly \bsc{Tang}\\
	\textit{Groupe B}\\
\end{flushleft}
\begin{flushright}
	\vspace{-3cm}
	\ifthenelse{\equal{\premierDestinataire}{}}{
	}
	{
		Pour \premierDestinataire(\rolePremierDestinataire)\\
	}
	\ifthenelse{\equal{\secondDestinaire}{}}{
	}
	{
		Pour \secondDestinaire(\roleSecondDestinaire)\\
	}
	\ifthenelse{\equal{\troisiemeDestinaire}{}}{
	}
	{
		Pour \troisiemeDestinaire(\roleTroisiemeDestinaire)\\
	}
	\ifthenelse{\equal{\quatriemeDestinaire}{}}{
	}
	{
		Pour \quatriemeDestinaire(\roleQuatriemeDestinaire)\\
	}
	\ifthenelse{\equal{\cinquiemeDestinaire}{}}{
	}
	{
		Pour \cinquiemeDestinaire(\roleCinquiereDestinaire)\\
	}
\end{flushright}
\vfill
\vspace{1cm}
\begin{flushleft}
\usefont{OT1}{ptm}{m}{n}
\huge \@title
\end{flushleft}
\par
\hrule height 4pt
\par
\begin{flushright}
\usefont{OT1}{phv}{m}{n}
\Large \@author
\par
\end{flushright}
\vspace{1cm}
\vfill
\vfill
\bas{}{\@location, le \@date}{}
}%
\cleardoublepage
}
\def\date#1{\def\@date{#1}}
\def\author#1{\def\@author{#1}}
\def\title#1{\def\@title{#1}}
\def\location#1{\def\@location{#1}}
\def\blurb#1{\def\@blurb{#1}}
\date{\today}
\author{}
\title{}
\location{Amiens}\blurb{}
\makeatother
\title{\titreDocument}
\author{Bibliothèque d'objets graphiques UML}

\location{Toulouse}
\blurb{%
Université Paul Sabatier -- Toulouse III\\
IUT A - Toulouse Rangueil\\
\textbf{Projet tuteuré}\\[1em]
}%



\pagestyle{fancy}

\definecolor{gris1}{gray}{0.40}
\definecolor{gris2}{gray}{0.55}
\definecolor{gris3}{gray}{0.65}
\definecolor{gris4}{gray}{0.50}
\definecolor{vert}{rgb}{0,0.4,0}
\definecolor{bleu1}{rgb}{0,0,0.8}
\definecolor{bleu2}{rgb}{0,0.2,0.6}
\definecolor{bleu3}{rgb}{0,0.2,0.2}


\lstdefinelanguage{algo}{%
   morekeywords={%
    %%% couleur 1
		importer, programme, glossaire, fonction, procedure, constante, type, 
	%%% IMPORT & Co.
		si, sinon, alors, fin, tantque, debut, faire, lorsque, fin lorsque, 
		declenche, declencher, enregistrement, tableau, retourne, retourner, =, 
		/=, <, >, traite,exception, 
	%%% types 
		Entier, Reel, Booleen, Caractere, Réél, Booléen, Caractère,
	%%% types 
		entree, maj, sortie,entrée,
	%%% types 
		et, ou, non,
	},
  sensitive=true,
  morecomment=[l]{--},
  morestring=[b]',
}

\lstset{language=algo,
    %%% BOUCLE, TEST & Co.
      emph={importer, programme, glossaire, fonction, procedure, constante, type},
      emphstyle=\color{bleu2},
    %%% IMPORT & Co.  
	emph={[2]
		si, sinon, alors, fin , tantque, debut, faire, lorsque, fin lorsque, 
		declencher, retourner, et, ou, non,enregistrement, retourner, retourne, 
		tableau, /=, <, =, >, traite,exception
	},
      emphstyle=[2]\color{bleu1},
    %%% FONCTIONS NUMERIQUES
      emph={[3]Entier, Reel, Booleen, Caractere, Booléen, Réél, Caractère},
      emphstyle=[3]\color{gris1},
    %%% FONCTIONS NUMERIQUES
      emph={[4]entree, maj, sortie, entrée},	
      emphstyle=[4]\color{gris1},
}
\lstdefinelanguage{wl}{%
   morekeywords={%
    %%% couleur 1
		importer, programme, glossaire, fonction, procedure, constante, type, 
	%%% IMPORT & Co.
		si, sinon, alors, fin, TANTQUE, tantque, FIN, PROCEDURE, debut, faire, lorsque, 
		fin lorsque, declenche, declencher, enregistrement, tableau, retourne, retourner, =, 
		/=, <, >, traite,exception, 
	%%% types 
		Entier, Reel, Booleen, Caractere, Réél, Booléen, Caractère,
	%%% types 
		entree, maj, sortie,entrée,
	%%% types 
		et, ou, non,
	},
  sensitive=true,
  morecomment=[l]{//},
  morestring=[b]',
}

\lstset{language=wl,
    %%% BOUCLE, TEST & Co.
      emph={importer, programme, glossaire, fonction, procedure, constante, type},
      emphstyle=\color{bleu2},
    %%% IMPORT & Co.  
	emph={[2]
		si, sinon, alors, fin , tantque, debut, faire, lorsque, fin lorsque, 
		declencher, retourner, et, ou, non,enregistrement, retourner, retourne, 
		tableau, /=, <, =, >, traite,exception
	},
      emphstyle=[2]\color{bleu1},
    %%% FONCTIONS NUMERIQUES
      emph={[3]Entier, Reel, Booleen, Caractere, Booléen, Réél, Caractère},
      emphstyle=[3]\color{gris1},
    %%% FONCTIONS NUMERIQUES
      emph={[4]entree, maj, sortie, entrée},	
      emphstyle=[4]\color{gris1},
}
\lstdefinelanguage{css}{%
   morekeywords={%
    %%% couleur 1
		background, image, repeat, position, index, color, border, font, 
		size, url, family, style, variant, weight, letter, spacing, line, 
		height, text, decoration, align, indent, transform, shadow, 
		background, image, repeat, position, index, color, border, font, 
		size, url, family, style, variant, weight, letter, spacing, line, 
		height, text, decoration, align, indent, transform, shadow, 
		vertical, align, white, space, word, spacing,attachment, width, 
		max, min, margin, padding, clip, direction, display, overflow,
		visibility, clear, float, top, right, bottom, left, list, type, 
		collapse, side, empty, cells, table, layout, cursor, marks, page, break,
		before, after, inside, orphans, windows, azimuth, after, before, cue, 
		elevation, pause, play, during, pitch, range, richness, spek, header, 
		numeral, punctuation, rate, stress, voice, volume,
	%%% types 
		left, right, bottom, top, none, center, solid, black, blue, red, green,
	},
  sensitive=true,
  sensitive=true,
  morecomment=[s]{/*}{*/},
  morestring=[b]',
}
\lstset{language=css,
    %%% BOUCLE, TEST & Co.
      emph={
		background, image, repeat, position, index, color, border, font, 
		size, url, family, style, variant, weight, letter, spacing, line, 
		height, text, decoration, align, indent, transform, shadow, 
		background, image, repeat, position, index, color, border, font, 
		size, url, family, style, variant, weight, letter, spacing, line, 
		height, text, decoration, align, indent, transform, shadow, 
		vertical, align, white, space, word, spacing,attachment, width, 
		max, min, margin, padding, clip, direction, display, overflow,
		visibility, clear, float, top, right, bottom, left, list, type, 
		collapse, side, empty, cells, table, layout, cursor, marks, page, break,
		before, after, inside, orphans, windows, azimuth, after, before, cue, 
		elevation, pause, play, during, pitch, range, richness, spek, header, 
		numeral, punctuation, rate, stress, voice, volume,
	  },
      emphstyle=\color{bleu2},
    %%% FONCTIONS NUMERIQUES
      emph={[3]
		left, right, bottom, top,none, solid, black, blue, green,
		  },
      emphstyle=[3]\color{bleu3},
    %%% FONCTIONS NUMERIQUES
}
\lstdefinelanguage{SQL}{%
   morekeywords={%
    %%% couleur 1
	INSERT, if, end, begin,UPDATE, DELETE, SET, WHERE, 
	CREATE, OR, REPLACE, AND, FROM, SELECT, VIEW, TRIGGER, AS, GROUP,
	BY, ORDER, after, for, each, then, else, of, on
	},
  sensitive=true,
  morecomment=[l]{--},
  morestring=[b]',
}

\lstset{language=SQL,
    %%% BOUCLE, TEST & Co.
      emph={INSERT, UPDATE, DELETE, WHERE, SET, GROUP, BY, ORDER},
      emphstyle=\color{bleu2},
    %%% IMPORT & Co.  
	emph={[2]
		if, end, begin, then, for, each, else, after, of, on
	},
      emphstyle=[2]\color{bleu1},
    %%% FONCTIONS NUMERIQUES
      emph={[3]Entier, Reel, Booleen, Caractere, Booléen, Réél, Caractère},
      emphstyle=[3]\color{gris1},
    %%% FONCTIONS NUMERIQUES
      emph={[4]entree, maj, sortie, entrée},	
      emphstyle=[4]\color{gris1},
}
\lstset{ % general style for listings 
   numbers=left 
   , literate={é}{{\'e}}1 {è}{{\`e}}1 {à}{{\`a}}1 {ê}{{\^e}}1 {É}{{\'E}}1 {ô}{{\^o}}1 {€}{{\euro}}1{°}{{$^{\circ}$}}1 {ç}{ {c}}1 {î}{{\^i}}1
	, extendedchars=\true
   , tabsize=2 
   , frame=single 
   , breaklines=true 
   , basicstyle=\ttfamily 
   , numberstyle=\tiny\ttfamily 
   , framexleftmargin=0mm 
   , xleftmargin=0mm 
   , captionpos=b 
	, keywordstyle=\color{bleu2}
	, commentstyle=\color{vert}
	, showstringspaces=false
	, extendedchars=true
	, mathescape=true
} 

\begin{document}
	\maketitle
	\newpage
	\tableofcontents
	\newpage
	LibUML est une 
	\glo{bibliothèque}{Bibliothèque}{Composant programmé dans un langage donné fournissant des méthodes permettant d'effectuer des tâches voulut} 
	d'objets graphiques représentant les différents éléments de modélisation de la norme 
	\glo{UML}{UML}{(Unified Modeling Language) Langage de modélisation graphique à base de pictogramme.  Il est apparu dans le monde du génie logiciel dans le cadre de la conception orientée objet. Ce langage est composé de différents diagrammes, allant du développement à la simple analyse des besoins.} 
	2\footnote{Unified Modelling Language}.
	Celle-ci à été développé dans le cadre des projets tuteurés à l'IUT\footnote{Institut Universitaire de Technologies} 'A' de Toulouse. 
	\begin{itemize}
		\item Antoine de \bsc{Roquemaurel}
		\item Mathieu \bsc{Soum}
		\item Geoffroy \bsc{Subias}
		\item Marie-Ly \bsc{Tag}
	\end{itemize}
	\vspace{15px}
	Nous l'avons développée en 
	\glo{Java}{Java}{Langage de programmation orienté objet moderne, il compile le programme pour ensuite l'exécuter sur une machine Java, ainsi le programme une fois compilé peut être exécuté sur différentes plateformes (Windows, Linux, Mac OS X, \ldots).} 
	et conçut comme une bibliothèque pouvant être utilisée dans des programmes Java comme composant. 
	Vous pouvez vous en servir pour développer un outil complet de modélisation UML par exemple.
	\paragraph{}	
	Cette bibliothèque permet de faire différentes choses dans le cadre de la conception UML, une fois la
	que vous aurez compris comment fonctionne la bibliothèque vous pourrez:
	\begin{itemize}
		\item Créer des éléments de modélisation 
			\begin{itemize}
				\item Acteur actif ou passif
				\item Traitement
				\item Cas d'utilisation
				\item Classe
			\end{itemize}
		\item Relier ses composants via différents types de flèches
			\begin{itemize}
				\item Agrégation
				\item Composition
				\item Association binavigable ou mononavigable
				\item Messages synchrones ou asynchrones
				\item Généralisation
			\end{itemize}
		\item Supprimer des éléments et flèches
		\item Modifier le contenu des éléments et flèches
		\item Redimensionner les éléments de modélisation
	\end{itemize}
	\newpage
	\chapter{Installation et téléchargement de la bibliothèque}
	\section{JGraphx}
		Notre bibliothèque UML dépend d'une bibliothèque externe, cette bibliothèque est JGraphx, vous devez donc posséder JGraphx 1.8 pour que notre bibliothèque puisse fonctionner, vous devez inclure celle-ci dans votre projet.\\
	$\rhd$ \url{http://telechargements.joohoo.fr/projet\_IUT/JGraphx-1.8.jar}\\\\
	\section{LibUML}
	Vous pouvez choisir de télécharger notre bibliothèque via un .jar ou si vous préférez les sources sont également disponibles.\\
	De même, il est nécessaire d'inclure soit les sources de la bibliothèque, soit le .jar dans le projet pour pouvoir utiliser notre bibliothèque.\\
	$\rhd$ \url{http://telechargements.joohoo.fr/projet\_IUT/libUML.jar}\\
	$\rhd$ \url{http://telechargements.joohoo.fr/projet\_IUT/libUML-src.zip}\\\\
	\section{Version projet Netbeans}
	Également, une version existe sous forme de projet Netbeans dans le cas où vous utilisez cet EDI\footnote{Environnement de Développement Intégré}.\\
	Ce projet contient les sources de la bibliothèque UML, la bibliothèque JGraphx, les Tests unitaires du projet et le code du démonstrateur.\\
	$\rhd$ \url{http://telechargements.joohoo.fr/projet\_IUT/libUML-projetNetbeans-1.8.zip}\\\\
	\chapter{Utilisation de libUML}
	\section{Le démonstrateur}
	Nous avons développé un démonstrateur afin que vous puissiez comprendre comment fonctionne la bibliothèque, et pour montrer ses possibilités.
	Ainsi, tout ce qu'il est possible de faire avec la bibliothèque sera présent dans le démonstrateur.\\
	Dans cette partie nous allons passer brièvement ses fonctionnalités, cependant, pour une meilleure
	compréhension, nous avons commentés le code du démonstrateur pour qu'il soit le plus simple possible et pour que vous ayez le moins 
	possible de vous référez à ce document. 
	
	Le démonstrateur est volontairement simple, il n'a pas pour but d'être lourd, mais uniquement de montrer les possibilités de la bibliothèque. 
	Il est disposé en trois parties:
	\begin{itemize}
		\item En haut, la barre d'outils, permettant de sélectionner l'élément graphique souhaité
		\item Au centre, ce que nous appelons le graphe, c'est ici qu'apparaîtrons les diagrammes UML
		\item À droite, un panneau contenant éventuellement un tableau avec les informations de la classe sélectionnée.
	\end{itemize}
	%%%
	%%% //TODO ici mettre un screen du démonstrateur.
	%%%
	\subsection{La barre d'outils}
	La barre d'outils contient tous les éléments graphiques que permet notre bibliothèque.
	\subsubsection{Element de modélisation}
	Il est actuellement possible de faire les éléments de modélisation suivants:
	\begin{itemize}
		\item Acteur actif
		\item Acteur passif
		\item Traitement
		\item Cas d'utilisation
		\item Classe
	\end{itemize}
	Le clic sur un bouton positionnera un élément de modélisation dans le graphe.\\
	Chacun des éléments de modélisation correspondent à une classe(Traitement, CasUtilisation,\ldots), qui hérite de la classe abstraite \texttt{ElementModelisation} 
	elle même héritant de la classe \texttt{ElementGraphique}. (Pour plus de détails, cf Annexe ?? page ??)% TODO

	\subsubsection{Liens}
	La deuxième partie de la barre d'outils contiens tous les liens qu'il est possible de faire.
	\begin{itemize}
		\item Généralisation
		\item Association
		\item Dépendance
		\item \ldots
	\end{itemize}
	Pour relier deux éléments, vous devez cliquer sur la flèche voulut, puis cliquer sur les deux éléments à relier. 
	Chacun des liens héritent de la classe abstraite \texttt{ElementGraphique}. (Pour plus de détails, cf Annexe ?? page ??).% TODO 
	Le clic sur les deux éléments aura pour effet de créer un lien, son constructeur accueillant la source et la destination du lien.
	
	\subsection{Le graphe}
	Le graphe est la partie central du démonstrateur, c'est dans celui-ci que tous les éléments graphiques s'affichent.
	\subsubsection{mxGraph}
	Le graphe est un objet de la classe \texttt{mxGraph}, pour pouvoir afficher les éléments graphiques de la bibliothèque 
	celui-ci est indispensable, si vous souhaitez afficher un élément graphique vous devrez donc instancier 
	un \texttt{mxGraph}, que vous devez ensuite faire passer à la création de chaque élément graphique via son constructeur, 
	en effet chaque élément graphique possèdent un graphe.
	%// TODO Cadre Exemples!!

	\subsubsection{Le diagramme}
	Le diagramme est un objet contenant la liste de tous les éléments graphiques présents dans le graphe, celui-ci peut vous permettre
	d'autoriser ou non la présence de certains éléments dans le graphe via la méthode \texttt{estAutorise}. Dans une approche de long
	terme, ce diagramme devrait pouvoir permettre d'exporter notre travail pour pouvoir le réutiliser plus tard, 
	en effet toutes les informations des éléments graphiques sont stockés dans le diagrammes (nom, position, taille, liens, \ldots).

	\subsubsection{Actions sur le graphe -- Événements}
	Pour pouvoir ajouter un \texttt{listener} sur le graphe, et ainsi pouvoir interagir avec des clics de souris, 
	il faut utiliser la Classe \texttt{mxGraphControl} qui possèdent une méthode \texttt{addMousseListener}.
	Ainsi, sur le démonstrateur, le clic droit sur un élément permet de faire des choses différentes en fonction de l'élément. \\
	\begin{itemize}
		\item Sur tout élément graphique, un clic droit permettra de le supprimer.
			\begin{itemize}
				\item Ce qui appellera à une super méthode \texttt{supprimer()}.
			\end{itemize}
		\item Sur un acteur (Actif ou Passif), un clic droit permet d'afficher ou non la ligne de vie
			\begin{itemize}
				\item Ce qui appellera une super méthode \texttt{afficherLigneDeVie(boolean)}.
			\end{itemize}
		\item Sur un traitement, le menu contextuel permet d'afficher la flèche ou non de début de séquence
			\begin{itemize}
				\item Ce qui appellera une méthode \texttt{setDebutSequence(boolean)}.
			\end{itemize}
	\end{itemize}
	%//TODO screen

	Lorsque vous sélectionnez un élément sur le graphe, si dans le code vous voulez récupérer l'élément graphique correspondant, 
	et ainsi appeler différentes méthodes, vous devez utilisez deux méthodes, une développée par JGraphx, une présente dans libUML. 
	\begin{itemize}
		\item \texttt{mxGraph.getSelectedCell()} qui va vous renvoyer une cellule, vous passez ensuite cette cellule en paramètre de la méthode suivante.
		\item \texttt{Diagramme.getElementGraphiqueViaCellule(mxCell)} qui va vous renvoyer un élément graphique correspondant à la cellule.
	\end{itemize}
	
	\subsection{Le panneau de droite}
		Le panneau de droite est un panneau servant à afficher des informations. Ainsi, lorsque vous cliquez sur une classe, un tableau s'affiche
		avec la liste des méthodes et des attributs de la classe concernée. Ce panneau à pour but de vous montrer la possibilité
		de récupérer les informations stockés et de les interpréter. Certaines informations sont présente dans une classe même
		si elle ne s'affiche pas dans le graphe, celle-ci on pour but de pouvoir être exploités si besoin, notamment avec de la génération
		de code (\texttt{static}, \texttt{final}, visibilité de la classe, \ldots).
	\section{Exemple de création d'un diagramme simple}
		Nous allons vous montrer comment fonctionne notre bibliothèque à l'aide d'un exemple, cet exemple est relativement simple
		et vient en complément du démonstrateur pour que vous puissiez comprendre rapidement la logique de la bibliothèque sans rentrer dans les détails
		du démonstrateur. Ainsi, nous allons réaliser le diagramme ci-dessous, en vous expliquant pas à pas la démarche qui est relativement simple.
		%//TODO screen du résultat de l'exemple.
		%//TODO exemple, à compiler tester ^_^
		\lstinputlisting[caption=Exemple de création d'un diagramme simple, language=Java]{exemple.java}
	\section{Documentation}
		L'exemple ci-dessus est succins pour apprendre à se servir de la bibliothèque, cependant il vous donne la logique, pour voir toutes les possibilités
		de la bibliothèque, vous pouvez voir le démonstrateur. 

		Également, la documentation sous forme de Javadoc est accessible ici:\\
	$\rhd$ \url{http://documentation.joohoo.fr/projet\_IUT/libUML/index.html}\\
	Grâce à celle-ci, vous pouvez savoir comment fonctionne et à quoi sert chaque méthode.

	\chapter{Poursuite de développement de la bibliothèque}				
	La bibliothèque utilise entièrement JGraphx pour dessiner les composants. Nous avons fait en sorte qu'un utilisateur voulant se servir de libUML
	n'ai nullement besoin de la connaissance de JGraphx, ainsi, il faut restreindre l'utilisation de JGraphx au strict minimum (\texttt{mxGraph}\ldots) quitte
	à redéfinir des fonctions appellant la fonction parente, ça à l'avantage d'intégrer la Javadoc de cette méthode à la documentation du projet.

	La bibliothèque JGraphx à le grand défaut de n'avoir qu'une toute petite communauté, ainsi, vous trouverez rarement des réponses à votre problème
	sur un forum, nous avons donc beaucoup utilisé la documentation (Javadoc et manuel), cette documentation est accessible ici: 
	%TODO \ldots
	
	\chapter{Conclusion du projet}
	
	\closeout\glossaireVar
	\appendix
	\chapter{Diagramme de classes}
	\chapter{Diagramme de paquetages}
	\chapter{Glossaire}
	\begin{sortedlist}
		\chapter {Glossaire}

	\end{sortedlist}
\end{document}

