\documentclass[12pt,a4paper,openany]{report}
\usepackage{lmodern}
\usepackage{xcolor}
\usepackage[utf8]{inputenc}
\usepackage[T1]{fontenc}
\usepackage[francais]{babel}
\usepackage[top=1.7cm, bottom=1.7cm, left=1.5cm, right=1.5cm]{geometry}
\usepackage{pdfpages}
\usepackage{listingsutf8}
\usepackage{fancyhdr}
\usepackage{multido}
\usepackage{amssymb}
\usepackage{tikz}
\usepackage{ifthen}
\usepackage{makeidx}
\usepackage[urlbordercolor={1 1 1}, linkbordercolor={1 1 1}, urlcolor=blue]{hyperref}

\newcommand{\headGauche}{}
\newcommand{\headCentre}{}
\newcommand{\headDroite}{}
%\newCommand{\footGauche}{} Université paul sabatier Toulouse III
\newcommand{\footCentre}{}
%\newCommand{\footDroite}{} Numéro de page
\newcommand{\premierDestinataire}{Monsieur Thierry Millan}
\newcommand{\rolePremierDestinataire}{Client}

\newcommand{\secondDestinaire}{Madame Caroline Kross}
\newcommand{\roleSecondDestinaire}{Tutrice}

\newcommand{\troisiemeDestinaire}{}
\newcommand{\roleTroisiemeDestinaire}{}

\newcommand{\quatriemeDestinaire}{}
\newcommand{\roleQuatriemeDestinaire}{}

\newcommand{\cinquiemeDestinaire}{}
\newcommand{\roleCinquiereDestinaire}{}
\newcommand{\titreDocument}{Installation et exploitation}



\date{\today}

\chead{\headCentre}
\rhead{\headDroite}
\lhead{\headGauche}
\makeindex
\lfoot{Université Paul Sabatier Toulouse III}
\rfoot{--~\thepage~--}
\cfoot{\footCentre}
\makeglossary
\makeatletter
\def\clap#1{\hbox to 0pt{\hss #1\hss}}%
\def\ligne#1{%
\hbox to \hsize{%
\vbox{\centering #1}}}%
\def\haut#1#2#3{%
\hbox to \hsize{%
\rlap{\vtop{\raggedright #1}}%
\hss
\clap{\vtop{\centering #2}}%
\hss
\llap{\vtop{\raggedleft #3}}}}%
\def\bas#1#2#3{%
\hbox to \hsize{%
\rlap{\vbox{\raggedright #1}}%
\hss \clap{\vbox{\centering #2}}%
\hss
\llap{\vbox{\raggedleft #3}}}}%
\def\maketitle{%
\thispagestyle{empty}\vbox to \vsize{%
\haut{}{\@blurb}{}
\begin{flushleft}
	\vspace{1cm}
	Antoine de \bsc{Roquemaurel}\\ 
	Mathieu \bsc{Soum}\\
	Geoffroy \bsc{Subias}\\
	Marie-Ly \bsc{Tang}\\
	\textit{Groupe B}\\
\end{flushleft}
\begin{flushright}
	\vspace{-3cm}
	\ifthenelse{\equal{\premierDestinataire}{}}{
	}
	{
		Pour \premierDestinataire(\rolePremierDestinataire)\\
	}
	\ifthenelse{\equal{\secondDestinaire}{}}{
	}
	{
		Pour \secondDestinaire(\roleSecondDestinaire)\\
	}
	\ifthenelse{\equal{\troisiemeDestinaire}{}}{
	}
	{
		Pour \troisiemeDestinaire(\roleTroisiemeDestinaire)\\
	}
	\ifthenelse{\equal{\quatriemeDestinaire}{}}{
	}
	{
		Pour \quatriemeDestinaire(\roleQuatriemeDestinaire)\\
	}
	\ifthenelse{\equal{\cinquiemeDestinaire}{}}{
	}
	{
		Pour \cinquiemeDestinaire(\roleCinquiereDestinaire)\\
	}
\end{flushright}
\vfill
\vspace{1cm}
\begin{flushleft}
\usefont{OT1}{ptm}{m}{n}
\huge \@title
\end{flushleft}
\par
\hrule height 4pt
\par
\begin{flushright}
\usefont{OT1}{phv}{m}{n}
\Large \@author
\par
\end{flushright}
\vspace{1cm}
\vfill
\vfill
\bas{}{\@location, le \@date}{}
}%
\cleardoublepage
}
\def\date#1{\def\@date{#1}}
\def\author#1{\def\@author{#1}}
\def\title#1{\def\@title{#1}}
\def\location#1{\def\@location{#1}}
\def\blurb#1{\def\@blurb{#1}}
\date{\today}
\author{}
\title{}
\location{Amiens}\blurb{}
\makeatother
\title{\titreDocument}
\author{Bibliothèque d'objets graphiques UML}

\location{Toulouse}
\blurb{%
Université Paul Sabatier -- Toulouse III\\
IUT A - Toulouse Rangueil\\
\textbf{Projet tuteuré}\\[1em]
}%



\pagestyle{fancy}

\begin{document}
	\maketitle
	\newpage
	\tableofcontents
	\vspace{20px}
	LibUML est une \glo{bibliothèque}{} d'objets graphiques représentant les différents éléments de modélisation de la norme \glo{UML}{
	(Unified Modeling Language) Langage de mod\'e lisation graphique à base de pictogramme. Il est apparu dans le monde du g\'enie logiciel dans le cadre de la conception orientée objet. Ce langage est composé de différents diagrammes, allant du développement à la simple analyse des besoins.} 2, celle-ci 
		est développée en Java et peut être utilisée dans des programmes Java comme composant. \\
		Cette bibliothèque peut servir à développer un outil complet de modélisation UML par exemple.
	\newpage
	\chapter{Installation de la bibliothèque}
	\section{Installation de la bibliothèque JGraphx}
	Notre bibliothèque UML dépend d'une bibliothèque externe, cette bibliothèque est JGraphx.
	\subsection{Téléchargement de JGraphx}
	Pour utiliser notre bibliothèque UML, vous devez tout d'abord télécharger la version 1.8 de JGraphx. Celle ci est accessible à l'adresse suviante:\\
	$\rhd$ \url{http://telechargements.joohoo.fr/projet\_IUT/JGraphx-1.8.jar}\\
	Une fois téléchargé, vous devez l'inclure à votre projet.
	\subsection{Téléchargement de notre bibliothèque UML}
	Vous devez ensuite télécharger notre bibliothèque\\
	$\rhd$ \url{http://telechargements.joohoo.fr/projet\_IUT/bibliothequeUML.jar}\\

	\chapter{Utilisation du démonstrateur}
	\chapter{Utilisation de la bibliothèque}
	\chapter{Poursuite de développement de la bibliothèque}				

	\closeout\glossaireVar
	\appendix
    \chapter {Glossaire}

\end{document}

