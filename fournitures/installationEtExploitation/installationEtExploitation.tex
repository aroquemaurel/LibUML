\documentclass[12pt,a4paper,openany]{report}
\usepackage{lmodern}
\usepackage{xcolor}
\usepackage[utf8]{inputenc}
\usepackage[T1]{fontenc}
\usepackage[francais]{babel}
\usepackage[top=1.7cm, bottom=1.7cm, left=1.5cm, right=1.5cm]{geometry}
\usepackage{pdfpages}
\usepackage{listingsutf8}
\usepackage{fancyhdr}
\usepackage{multido}
\usepackage{amssymb}
\usepackage{tikz}
\usepackage{ifthen}
\usepackage{makeidx}
\usepackage[urlbordercolor={1 1 1}, linkbordercolor={1 1 1}, urlcolor=blue]{hyperref}

\newcommand{\footCentre}{}
\newcommand{\premierDestinataire}{Monsieur Thierry Millan}
\newcommand{\rolePremierDestinataire}{Client}

\newcommand{\secondDestinaire}{Madame Caroline Kross}
\newcommand{\roleSecondDestinaire}{Tutrice}

\newcommand{\troisiemeDestinaire}{}
\newcommand{\roleTroisiemeDestinaire}{}

\newcommand{\quatriemeDestinaire}{}
\newcommand{\roleQuatriemeDestinaire}{}

\newcommand{\cinquiemeDestinaire}{}
\newcommand{\roleCinquiereDestinaire}{}

\newcommand{\titreDocument}{Installation et exploitation}



\date{\today}

\chead{\headCentre}
\rhead{\headDroite}
\lhead{\headGauche}
\makeindex
\lfoot{Université Paul Sabatier Toulouse III}
\rfoot{--~\thepage~--}
\cfoot{\footCentre}
\makeglossary
\makeatletter
\def\clap#1{\hbox to 0pt{\hss #1\hss}}%
\def\ligne#1{%
\hbox to \hsize{%
\vbox{\centering #1}}}%
\def\haut#1#2#3{%
\hbox to \hsize{%
\rlap{\vtop{\raggedright #1}}%
\hss
\clap{\vtop{\centering #2}}%
\hss
\llap{\vtop{\raggedleft #3}}}}%
\def\bas#1#2#3{%
\hbox to \hsize{%
\rlap{\vbox{\raggedright #1}}%
\hss \clap{\vbox{\centering #2}}%
\hss
\llap{\vbox{\raggedleft #3}}}}%
\def\maketitle{%
\thispagestyle{empty}\vbox to \vsize{%
\haut{}{\@blurb}{}
\begin{flushleft}
	\vspace{1cm}
	Antoine de \bsc{Roquemaurel}\\ 
	Mathieu \bsc{Soum}\\
	Geoffroy \bsc{Subias}\\
	Marie-Ly \bsc{Tang}\\
	\textit{Groupe B}\\
\end{flushleft}
\begin{flushright}
	\vspace{-3cm}
	\ifthenelse{\equal{\premierDestinataire}{}}{
	}
	{
		Pour \premierDestinataire(\rolePremierDestinataire)\\
	}
	\ifthenelse{\equal{\secondDestinaire}{}}{
	}
	{
		Pour \secondDestinaire(\roleSecondDestinaire)\\
	}
	\ifthenelse{\equal{\troisiemeDestinaire}{}}{
	}
	{
		Pour \troisiemeDestinaire(\roleTroisiemeDestinaire)\\
	}
	\ifthenelse{\equal{\quatriemeDestinaire}{}}{
	}
	{
		Pour \quatriemeDestinaire(\roleQuatriemeDestinaire)\\
	}
	\ifthenelse{\equal{\cinquiemeDestinaire}{}}{
	}
	{
		Pour \cinquiemeDestinaire(\roleCinquiereDestinaire)\\
	}
\end{flushright}
\vfill
\vspace{1cm}
\begin{flushleft}
\usefont{OT1}{ptm}{m}{n}
\huge \@title
\end{flushleft}
\par
\hrule height 4pt
\par
\begin{flushright}
\usefont{OT1}{phv}{m}{n}
\Large \@author
\par
\end{flushright}
\vspace{1cm}
\vfill
\vfill
\bas{}{\@location, le \@date}{}
}%
\cleardoublepage
}
\def\date#1{\def\@date{#1}}
\def\author#1{\def\@author{#1}}
\def\title#1{\def\@title{#1}}
\def\location#1{\def\@location{#1}}
\def\blurb#1{\def\@blurb{#1}}
\date{\today}
\author{}
\title{}
\location{Amiens}\blurb{}
\makeatother
\title{\titreDocument}
\author{Bibliothèque d'objets graphiques UML}

\location{Toulouse}
\blurb{%
Université Paul Sabatier -- Toulouse III\\
IUT A - Toulouse Rangueil\\
\textbf{Projet tuteuré}\\[1em]
}%



\pagestyle{fancy}

\begin{document}
	\maketitle
	\newpage
	\tableofcontents
	\vspace{20px}
	LibUML est une 
	\glo{bibliothèque}{Bibliothèque}{Composant programmé dans un langage donné fournissant des méthodes permettant d'effectuer des tâches voulut} 
	d'objets graphiques représentant les différents éléments de modélisation de la norme 
	\glo{UML}{UML}{(Unified Modeling Language) Langage de modélisation graphique à base de pictogramme.  Il est apparu dans le monde du génie logiciel dans le cadre de la conception orientée objet. Ce langage est composé de différents diagrammes, allant du développement à la simple analyse des besoins.} 
	2, celle-ci est développée en 
	\glo{Java}{Java}{Langage de programmation orienté objet moderne, il compile le programme pour ensuite l'exécuter sur une machine Java, ainsi le programme une fois compilé peut être exécuté sur différentes plateformes (Windows, Linux, Mac OS X, \ldots).} 
	et peut être utilisée dans des programmes Java comme composant. \\
	Cette bibliothèque peut servir à développer un outil complet de modélisation UML par exemple.
	\newpage
	\chapter{Installation de la bibliothèque}
		Notre bibliothèque UML dépend d'une bibliothèque externe, cette bibliothèque est JGraphx, vous devez donc posséder JGraphx 1.8 pour que notre bibliothèque puisse fonctionner. \\
	$\rhd$ \url{http://telechargements.joohoo.fr/projet\_IUT/JGraphx-1.8.jar}\\\\
	Vous pouvez avoir la version jar de notre bibliothèque UML, mais également les sources sont disponibles aux adresses suivantes.\\
	$\rhd$ \url{http://telechargements.joohoo.fr/projet\_IUT/libUML.jar}\\
	$\rhd$ \url{http://telechargements.joohoo.fr/projet\_IUT/libUML-src.zip}\\\\
	Si vous utilisez l'EDI\footnote{Environnement de Développement Intégré} Netbeans, il est possible de télécharger le projet complet(contenant JGraphx, les sources de la 
	bibliothèque, les tests unitaires et le démonstrateur), accessible ici:\\
	$\rhd$ \url{http://telechargements.joohoo.fr/projet\_IUT/libUML-projetNetbeans-1.8.zip}\\\\
	\chapter{Le démonstrateur}
	Nous avons développé un démonstrateur afin que vous puissiez comprendre comment fonctionne la bibliothèque, et pour montrer ses possibilités.
	Ainsi, tout ce qu'il est possible de faire avec la bibliothèque sera présent dans le démonstrateur.\\
	Dans cette partie nous allons passer brièvement ses fonctionnalités, et comment cela fonctionne, cependant, pour une meilleure
	compréhension, nous avons commentés le code du démonstrateur pour qu'il soit le plus simple possible. 
	
	Le démonstrateur est volontairement simple, il n'a pas pour but d'être lourd, mais uniquement de montrer les possibilités de la bibliothèque. 
	Il est disposé en trois parties:
	\begin{itemize}
		\item En haut, la barre d'outils, permettant de selectionner l'élément graphique souhaité
		\item Au centre, ce que nous appellons le graph, c'est ici qu'apparaîtrons les diagrammes UML
		\item À droite, un panneau contenant éventuellement un tableau avec les informations de la classe séléctionnée.
	\end{itemize}
	%%%
	%%% //TODO ici mettre un screen du démonstrateur.
	%%%
	\section{La barre d'outils}
	La barre d'outils contient tous les éléments graphiques que permet notre bibliothèque.
	\subsection{Element de modélisation}
	Il est actuellement possible de faire les éléments de modélisation suivants:
	\begin{itemize}
		\item Acteur actif
		\item Acteur passif
		\item Traitement
		\item Cas d'utilisation
		\item Classe
	\end{itemize}
	Le clic sur un bouton positionnera un élément de modélisation dans le graph.
	Chacun des éléments de modélisation correspondent à une classe,  qui hérite de la classe abstraite \policeCode{ElementModelisation} qui héritent
	elle même de la classe \policeCode{ElementGraphique}. (Pour plus de détails, cf Annexe ?? page ??)% TODO

	\subsection{Liens}
	La deuxième partie de la barre d'outils contiens tous les liens qu'il est possible de faire.
	\begin{itemize}
		\item Généralisation
		\item Association
		\item Dépendance
		\item \ldots
	\end{itemize}
	Pour relier deux éléments, vous devez cliquer sur la flèche voulut, puis cliquer sur les deux éléments à relier. 
	Chacun des liens héritent de la classe abstraite ElementGraphique. (Pour plus de détails, cf Annexe ?? page ??).% TODO 
	Le clic sur les deux éléments aura pour effet de créer un lien, son constructeur accueillant la source et la destination du lien.
	
	\section{Le graphe}
	Le graphe est la partie central du démonstrateur, c'est dans celui-ci que tous les éléments graphiques s'affichent.

	\subsection{mxGraph}
	Pour pouvoir afficher les éléments graphiques de la bibliothèque celui-ci est indispensable, si vous souhaitez
	afficher un élément graphique vous devrez donc instancier un \policeCode{mxGraph}, que vous devze ensuite faire passer
	à la création de chaque élément graphique via son constructeur, en effet chaque élément graphique possèdent un graphe.

	\subsection{Le diagramme}
	Le diagramme est un objet contenant la liste de tous les éléments graphiques présents dans le graph, celui-ci peut vous permettre
	d'autoriser ou non la présence de certains éléments dans le graph via la méthode \policeCode{estAutorise}. Dans une approche de long
	terme, ce diagramme devrait pouvoir permettre d'exporter notre travail pour pouvoir le réutiliser plus tard, 
	en effet toutes les informations des éléments graphiques sont stockés dans le diagrammes (nom, position, taille, liens, \ldots).

	\subsection{Actions sur le graphe -- Événements}
	Pour pouvoir ajouter un \policeCode{listener} sur le graphe, et ainsi pouvoir intéragire avec des clics de souris, 
	il faut utiliser la Classe \policeCode{mxGraphControl} qui possèdent une méthode \policeCode{addMousseListener}.
	Ainsi, sur le démonstrateur, le clic droit sur un élément permet de faire des choses différentes en fonction de l'élément. \\
	\begin{itemize}
		\item Sur tout élément graphique, un clic droit permettra de le supprimer.
			\begin{itemize}
				\item Ce qui appellera à une superMethode \policeCode{supprimer()}.
			\end{itemize}
		\item Sur un acteur (Actif ou Passif), un clic droit permet d'afficher ou non la ligne de vie
			\begin{itemize}
				\item Ce qui appellera une superMethode \policeCode{afficherLigneDeVie(boolean)}.
			\end{itemize}
		\item Sur un traitement, le menu contextuel permet d'afficher la flèche ou non de début de séquence
			\begin{itemize}
				\item Ce qui appellera une méthode \policeCode{setDebutSequence(boolean)}.
			\end{itemize}
	\end{itemize}

	Lorsque vous sélectionnez un élément sur le graphe, si dans le code vous voulez récupérer l'élément graphique correspondant, 
	et ainsi appeler différentes méthodes, vous devez utilisez deux méthodes, une développée par JGraphx, une présente dans libUML. 
	\begin{itemize}
		\item \policeCode{mxGraph.getSelectedCell()} qui va vous renvoyer une cellule, vous passez ensuite cette cellule en paramètre de la méthode suivante.
		\item \policeCode{Diagramme.getElementGraphiqueViaCellule(mxCell)} qui va vous renvoyer un élément graphique correspondant à la cellule.
	\end{itemize}
	
	\section{Le panneau de droite}

	
	\chapter{Poursuite de développement de la bibliothèque}				

	\closeout\glossaireVar
	\appendix
	\begin{sortedlist}
    \chapter {Glossaire}

	\end{sortedlist}
\end{document}

